\documentclass[11pt]{article}
\usepackage{amsmath, amssymb, amsfonts, amsthm, mathtools,mathrsfs}
\usepackage{thmtools}
\usepackage[utf8]{inputenc}
\usepackage[inline]{enumitem}
\usepackage[colorlinks=true]{hyperref}
\usepackage{multicol}
\usepackage{tikz}
\usetikzlibrary{decorations.markings}
\usetikzlibrary{arrows.meta}
\usepackage{witharrows}
\usepackage[useregional, showdow]{datetime2}
\usepackage{physics}
\DTMlangsetup[en-GB]{abbr}
\usepackage{xcolor}
\usepackage[normalem]{ulem}


\setlength\parindent{0pt}
\usepackage{parskip}

\def\D{\mathrm{d}}
\newcommand*{\thead}[1]{\multicolumn{1}{c}{\bfseries #1}}
\renewcommand{\arraystretch}{1}

\usepackage[framemethod=tikz]{mdframed}
\mdfdefinestyle{theoremstyle}{%
	% linecolor=gray,linewidth=1pt,%
	% frametitlerule=true,%
	frametitlebackgroundcolor=white,
	% backgroundcolor=  gray!20,	
	bottomline=false, topline=false, rightline=false, leftline=true,
	innerlinewidth=0.7pt, outerlinewidth=0.7pt, middlelinewidth=2pt, middlelinecolor=white, %
	innerleftmargin=6pt,
	% innertopmargin=-1pt,
	skipabove=10pt,
	% fontcolor=blue,
	% innerbottommargin=-0.5pt,
}
\mdtheorem[style=theoremstyle]{defn}[thm]{Definition}
\mdtheorem[style=theoremstyle]{lem}[thm]{Lemma}
\mdtheorem[style=theoremstyle]{thm}{Theorem}
\mdtheorem[style=theoremstyle]{claim}[thm]{Claim.}

\newcommand*{\doublerule}{\hrule width \hsize height 1pt \kern 0.5mm \hrule width \hsize height 2pt}
\newcommand{\doublerulefill}{\leavevmode\leaders\vbox{\hrule width .1pt\kern1pt\hrule}\hfill\kern0pt}
\def\ddfrac#1#2{\displaystyle\frac{\displaystyle #1}{\displaystyle #2}}


%\newcommand{\Res}{\operatorname{Res}}

\theoremstyle{definition}
% \numberwithin{thm}{section}
% \newtheorem{lem}[thm]{Lemma}
% \newtheorem{defn}[thm]{Definition}
% \newtheorem{prop}[thm]{Proposition}
% \newtheorem{cor}[thm]{Corollary}



\let\emptyset\varnothing

\usepackage{titlesec}
\titleformat{\section}[block]{\Large\filcenter\bfseries}{\S\thesection.}{0.25cm}{\Large}
\titleformat{\subsection}[block]{\large\bfseries\sffamily}{\S\S\thesubsection.}{0.2cm}{\large}

\usepackage[a4paper]{geometry}
\usepackage{lipsum}
\usepackage{xcolor,cancel}

\usepackage{cleveref}
\crefname{thm}{Theorem}{Theorems}
\crefname{lem}{Lemma}{Lemmas}
\crefname{defn}{Definition}{Definitions}
\crefname{prop}{Proposition}{Propositions}
\crefname{cor}{Corollary}{Corollaries}
\crefname{equation}{}{}


\usepackage{mdframed}
\newenvironment{blockquote}
{\begin{mdframed}[skipabove=0pt, skipbelow=0pt, innertopmargin=4pt, innerbottommargin=4pt, bottomline=false,topline=false,rightline=false, linewidth=2pt]}
{\end{mdframed}}
\newenvironment{soln}{\begin{proof}[Solution]}{\end{proof}}



\title{MA 205: Complex Analysis\\Tutorial Solutions}
\author{Ishan Kapnadak}
\date{Autumn Semester 2021-22\\~\\Updated on: \textcolor{blue}{\DTMToday}}

\begin{document}
\tikzset{lab dis/.store in=\LabDis,
  lab dis=-0.4,
  ->-/.style args={at #1 with label #2}{decoration={
    markings,
    mark=at position #1 with {\arrow{>}; \node at (0,\LabDis) {#2};}},postaction={decorate}},
  -<-/.style args={at #1 with label #2}{decoration={
    markings,
    mark=at position #1 with {\arrow{<}; \node at (0,\LabDis)
    {#2};}},postaction={decorate}},
  -*-/.style args={at #1 with label #2}{decoration={
    markings,
    mark=at position #1 with {{\fill (0,0) circle (1.5pt);} \node at (0,\LabDis)
    {#2};}},postaction={decorate}},
  }
\maketitle
\tableofcontents

\medskip

\underline{Note}: Many of these solutions are either inspired by, or in some cases directly taken from Aryaman Maithani's \href{https://aryamanmaithani.github.io/ma-205-tut/tut-solutions.pdf}{tutorial solutions} for last year's offering of this course. 

\newpage\section{Week 1}
\begin{center}
	3rd August, 2021
\end{center}

\underline{Notation}: We use $\mathbb{C}[x]$ to denote the set of all polynomials in $x$ with complex coefficients. $\mathbb{R}[x]$ is defined similarly. 

\begin{enumerate}[leftmargin=*]
    \itemsep0.5em
    \item Show that a real polynomial that is irreducible has degree at most two, i.e, if 
    \[
        f(x) = a_0 + a_1x + \cdots + a_n x^n , a_i \in \mathbb{R},
    \]
    then there are non-constant real polynomials $g$ and $h$ such that $f(x) = g(x) h(x)$ if $n \geq 3$. \textcolor{blue}{($a_n \neq 0$, of course)}
    
    \begin{soln}
        We consider two cases. First, suppose $f(x) \in \mathbb{R}[x]$ has a real root, $x_0$, and let $h(x) \vcentcolon= (x-x_0)$. Since $x_0 \in \mathbb{R}$, $h(x) \in \mathbb{R}[x]$. Moreover, we can write 
        \[
            f(x) = g(x) h(x)
        \]
        for some $g(x) \in \mathbb{R}[x]$. (Why must $g$ be a real polynomial?) Also, since $\deg f(x) \geq 3$ and $\deg h(x) = 1$, we have that $\deg g(x) \geq 2$. Thus, $g$ and $h$ are two non-constant real polynomials satisfying $f(x) = g(x) h(x)$.
        
        Now, suppose that $f(x)$ has no real root. We may also view $f(x)$ as a polynomial in $\mathbb{C}[x]$. By FTA, we know that $f(x)$ has a complex root $x_0 \in \mathbb{C}$. By assumption, we have that $x_0 \notin \mathbb{R}$, and thus $x_0 \neq \overline{x_0}$.
        
        \medskip
        
        \begin{blockquote}
			\textbf{Claim.} $f(\overline{x_0}) = 0.$
			\begin{proof} 
				We have
				\[\begin{WithArrows}[displaystyle]
			    f(\overline{x_0}) &= a_0 + a_1\overline{x_0} + \cdots + a_n(\overline{x_0})^n \Arrow{\textcolor{blue}{$\overline{z^n} = \bar{z}^n$}}\\
					&= a_0 + a_1\overline{x_0} + \cdots + a_n\overline{x_0^n} \Arrow{\textcolor{blue}{$a_i \in \mathbb{R}  \text{ \emph{and thus,} } a_i = \overline{a_i}$}}\\
					&= \overline{a_0} + \overline{a_1}\;\overline{x_0} + \cdots + \overline{a_n}\overline{x_0^n} \Arrow{\textcolor{blue}{$\overline{z_1z_2 + z_3} = \overline{z_1}\;\overline{z_2} + \overline{z_3}$}}\\
					&= \overline{f(x_0)}\\
					&= \bar{0} \\
					&= 0.
			  \end{WithArrows}\]
			\end{proof}
		\end{blockquote}
		
		Thus, $x_0$ and $\overline{x_0}$ are two distinct roots of $f(x)$. Define $g(x) \vcentcolon= (x-x_0)(x-\overline{x_0})$. A priori, we have $g(x) \in \mathbb{C}[x]$. However, note that
		\[
		    (x-x_0)(x-\overline{x_0}) = x^2 - (2\mathfrak{R}x_0)x + \abs{x_0}^2 \in \mathbb{R}[x].
		\]
		Thus, $g(x)$ is in fact a real polynomial. Since $x_0$ and $\overline{x_0}$ are distinct, we see that $g(x)$ divides $f(x)$ in $\mathbb{C}[x]$. (Why?) Thus,
		\[
		    f(x) = g(x) h(x)
		\]
		for some $h(x) \in \mathbb{C}[x]$. Again, since $f(x)$ and $g(x)$ are both real polynomials, so is $f(x)$. Moreover, since $\deg f(x) \geq 3$ and $\deg g(x) = 2$, we have $\deg h(x) \geq 1$, and we are done.

    \end{soln}
    
    \item Show that a non-constant polynomial $f(z_1, z_2)$ in complex variables $z_1$ and $z_2$ with complex coefficients, has infinitely many roots in $\mathbb{C}^2$.
    
    \begin{soln}
        Before we prove this, we first prove the following useful Lemma.
        
        \medskip
        
        \begin{blockquote}
			\textbf{Lemma.} A complex polynomial of degree $n$ has exactly $n$ roots, counted with multiplicity. In particular, all complex polynomials have finitely many roots. 
			\begin{proof} 
				Let $f(x) \in \mathbb{C}[x]$ be a polynomial of degree $n$. We prove this via induction on $n$. When $n = 1$, $f(x) = a_0 + a_1x$ for some $a_0,a_1 \in \mathbb{C}$ with $a_1 \neq 0$. We have
				\begin{align*}
				    f(x) &= 0 \\
				    \iff a_0 + a_1 x &= 0 \\
				    \iff a_1x &= -a_0 \\
				    \iff x &= -\frac{a_0}{a_1}.
				\end{align*}
				Thus, $f(x)$ has exactly $1$ root. 
				
				We now assume that an $n$-degree polynomial $g(x) \in \mathbb{C}[x]$ has exactly $n$ roots (counted with multiplicity). Let $f(x) \in \mathbb{C}[x]$ have degree $n+1$. By FTA, $f(x)$ has a root $x_0 \in \mathbb{C}$. We may thus write
				\[
				    f(x) = (x-x_0)g(x),
				\]
				for some $n$-degree polynomial $g(x) \in \mathbb{C}[x]$. Now, we have
				\[
				    f(x) = 0 \iff x = x_0 \text{ or } g(x) = 0.
				\]
				By assumption, the latter happens for exactly $n$ values of $x$. Thus, $f(x)$ has exactly $n+1$ roots counted with multiplicity. The second statement follows from the fact that any polynomial has finite degree.
			\end{proof}
		\end{blockquote}
        
        Since $f(z_1, z_2)$ is non-constant at least one of $z_1$ or $z_2$ must ``appear'' in $f(z_1, z_2)$. Without loss of generality, suppose that $z_2$ appears in $f(z_1, z_2)$. We may write
        \[
            f(z_1, z_2) = \sum_{k=0}^n f_k(z_1) \cdot z_2^k
        \]
        where $n \geq 1$ and $f_k(z_1) \in \mathbb{C}[z_1]$. Moreover, $f_n \neq 0$, and thus, $f_n(z_1)$ has only finitely many roots (possibly zero). Thus, there are infinitely many $\alpha \in \mathbb{C}$ such that $f_n(\alpha) \neq 0$. Since, $n \geq 1$, we have that $f(\alpha, z_2) \in \mathbb{C}[z_2]$ is non-constant for all these infinitely many $\alpha$. By FTA, for each such $\alpha$, there exists $\beta \in \mathbb{C}$ such that $f(\alpha, \beta) = 0$. Thus, there are infinitely many roots of $f(z_1, z_2)$ in $\mathbb{C}^2$ (since it contains all these pairs $(\alpha, \beta)$ as $\alpha$ takes on infinitely many values).
    \end{soln}
    
    \item Show that the complex plane minus a countable set is path-connected.
    
    \begin{soln}
        Let $S \subset \mathbb{C}$ be countable. We must show that $\mathbb{C} \setminus S$ is path-connected. Let $z_1, z_2 \in \mathbb{C} \setminus S$ and $z_1 \neq z_2$. Let $f$ be the line segment joining $z_1$ to $z_2$, and let $g$ be the circular arc joining $z_1$ to $z_2$. For every $\lambda \in [0,1]$, we define
        \[
            \sigma_{\lambda}(t) \vcentcolon= \lambda f(t) + (1-\lambda)g(t) \quad \forall \, t \in [0,1]
        \]
        
        \begin{blockquote}
			\textbf{Claim.} \begin{enumerate}
			    \item $\sigma_{\lambda}$ is a path in $\mathbb{C}$,
			    \item $\sigma_{\lambda}(0) = z_1$ and $\sigma_{\lambda}(1) = z_2$ for all $\lambda \in [0,1]$, and
			    \item if $\lambda_1 \neq \lambda_2$ and $t \in (0,1)$, then $\sigma_{\lambda_1}(t) \neq \sigma_{\lambda_2}(t)$.
			\end{enumerate} 
			\begin{proof} 
				We leave the proof for (a) and (b) as simple exercises. To show (c), we first note that for $t \in (0,1)$, $f(t) \neq g(t)$. Now, let $\lambda_1, \lambda_2 \in [0,1]$ with $\lambda_1 \neq \lambda_2$. Suppose $\sigma_{\lambda_1}(t) = \sigma_{\lambda_2}(t)$. We then have
				\begin{align*}
				    &\lambda_1 f(t) + (1-\lambda_1) g(t) = \lambda_2 f(t) + (1-\lambda_2) g(t) \\ \implies &(\lambda_1 - \lambda_2) f(t) = (\lambda_1 - \lambda_2) g(t).
				\end{align*}
				Since $\lambda_1 \neq \lambda_2$, we get $f(t) = g(t)$, a contradiction. Intuitively, this means that the images of all these paths are disjoint, barring the start and end points. 
			\end{proof}

		\end{blockquote}
		
		Since $[0,1]$ is uncountable (we assume this without proof), and the images are disjoint (by claim (c)), we have that the set $\left\{ \sigma_{\lambda} \mid \lambda \in [0,1] \right\}$ is uncountable. Since the set $S$ is only countable, there exists some $\lambda_0 \in [0,1]$ such that $\sigma_{\lambda_0}(t) \notin S$ for all $t \in [0,1]$. In other words, $\sigma_{\lambda_0}$ is a path in $\mathbb{C} \setminus S$ starting at $z_1$ and ending at $z_2$. Since $z_1,z_2$ were arbitrary, we are done.
        
    \end{soln}
    
    \item Check for real differentiability and holomorphicity:
    
    \begin{enumerate}
        \item $f(z) = c$
        \item $f(z) = z$
        \item $f(z) = z^n$, $n \in \mathbb{Z}$
        \item $f(z) = \mathfrak{R}z$
        \item $f(z) = \abs{z}$
        \item $f(z) = \abs{z}^2$
        \item $f(z) = \overline{z}$
        \item $f(z) = \begin{cases}
            \frac{z}{\overline{z}} & \text{if } z \neq 0 \\
            0 & \text{if } z = 0
        \end{cases}$
    \end{enumerate}
    \begin{soln} Some of these are trivial and hence omitted.
    \begin{enumerate}
        \item Real differentiable and holomorphic.
        \item Real differentiable and holomorphic.
        \item For $n \geq 0$, real differentiable and holomorphic. Since holomorphicity implies real differentiability, we only check for holomorphicity. Let $z_0 \in \mathbb{C}$ be arbitrary.  We must check for the existence of the following limit:
        \[
            \lim_{z \to z_0} \frac{f(z) - f(z_0)}{z-z_0}.
        \]
        For $z \neq z_0$, we know that
        \[
            \frac{z^n - z_0^n}{z-z_0} = \sum_{k=0}^{n-1} z^k z_0^{n-1-k}.
        \]
        Since the limit of the RHS exists as $z \to z_0$, we are done.
        
        \medskip
        
        For $n < 0$, the function is defined on $\mathbb{C} \setminus \{0\}$. On $\mathbb{C} \setminus \{0\}$, $f(z)$ is non-zero. Thus, $\frac{1}{f}$ is holomorphic on $\mathbb{C} \setminus \{0\}$ by the first case since $\frac{1}{f(z)} = z^{-n}$ and $-n > 0$. Thus, $f(z)$ is holomorphic on $\mathbb{C} \setminus \{0\}$.
        
        \item Real differentiable but not holomorphic. We may write $f$ as
        \[
            f(x+\iota y) = x + 0\iota.
        \]
        Thus, $u(x,y) = x$ and $v(x,y) = 0$. $f$ is clearly real differentiable since all the partial derivatives exist everywhere and are continuous. However, since $u_x(x_0,y_0) = 1$ and $v_y(x_0,y_0) = 0$ for all $(x_0,y_0) \in \mathbb{R}^2$, the CR equations do not hold. Hence, $f$ is complex differentiable nowhere, and thus, not holomorphic.
        
        \item $\abs{z}$ is real differentiable precisely on $\mathbb{C} \setminus \{0\}$ and complex differentiable nowhere. We may write
        \[
            f(x+\iota y) = \sqrt{x^2 + y^2} + 0\iota
        \]
        giving us $u(x,y) = \sqrt{x^2 + y^2}$, and $v(x,y) = 0$. On $\mathbb{R}^2\setminus\{(0,0)\}$, all partial derivatives exist and are continuous, whereas $u_x$ and $u_y$ fail to exist at $(0,0)$. Thus, $f(z)$ is real differentiable on $\mathbb{C} \setminus \{0\}$. Moreover, this shows that $f(z)$ is not complex differentiable at $0$ since it's not even real differentiable there. Everywhere else, $v_x = v_y = 0$, but at least one of $u_x, u_y$ is non-zero, violating the CR equations. Thus, $f(z)$ is complex differentiable nowhere.
        
        \item $\abs{z}^2$ is real differentiable everywhere and complex differentiable precisely at $0$. As a result, it is holomorphic nowhere. As before, we have $u(x,y) = x^2 + y^2$, and $v(x,y) = 0$. Since all partial derivatives exist everywhere and are continuous, $f(z)$ is real differentiable everywhere. Note that
        \begin{align*}
            u_x(x,y) = 2x &\quad u_y(x,y) = 2y \\
            v_x(x,y) = 0 &\quad v_y(x,y) = 0
        \end{align*}
        Thus, the CR equations hold precisely at $0$.
        
        \item For $f(z) = \overline{z}$, we may write
        \[
            f(x+\iota y) = x - \iota y,
        \]
        which gives us $u(x,y) = x$ and $v(x,y) = -y$. Since all partials exist everywhere and are continuous, $f(z)$ is real differentiable everywhere. However, note that
        \begin{align*}
            u_x(x,y) = 1 &\quad u_y(x,y) = 0 \\
            v_x(x,y) = 0 &\quad v_y(x,y) = -1
        \end{align*}
        Since $u_x(x,y) \neq v_y(x,y)$ for all $(x,y) \in \mathbb{R}^2$, we see that the CR equations do not hold anywhere and $f(z)$ is complex differentiable nowhere.
        
        \item $f(z)$ is real differentiable precisely on $\mathbb{C} \setminus \{ 0\}$, and complex differentiable nowhere. We may multiply and divide by $\overline{z}$ to obtain
        \[
            u(x,y) = \frac{x^2-y^2}{x^2+y^2} \quad \text{and} \quad v(x,y) = \frac{2xy}{x^2+y^2}
        \]
        for $(x,y) \neq (0,0)$, and $u(0,0) = v(0,0) = 0$. Since $u$ and $v$ are not continuous at $(0,0)$ (recall MA109), neither is $f$. Hence, $f$ is neither real differentiable, nor complex differentiable at $0 \in \mathbb{C}$. At all other points, all partials exist and are continuous. Hence, $f$ is real differentiable there. However, one may explicitly compute those partial derivatives and verify that the CR equations hold nowhere. Thus, $f$ is complex differentiable nowhere.
        
    \end{enumerate}
\end{soln}
\end{enumerate}



\end{document}
